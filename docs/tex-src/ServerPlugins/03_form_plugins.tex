\section{Form Plug-Ins}
	\subsection{Overview}
		\begin{table}[ht]
			\centering
			\begin{tabular}{@{}lr@{}}
			\toprule
			\textbf{Required superclass} & \textit{FormServerPlugin}
\\
			\textbf{Configurable Sub-URLs} & \textit{no} \\
			\textbf{Requires JavaScript} & \textit{no} \\
			\textbf{Methods to implement} & \textit{void
performFormAction( RequestFacade req , } \\
			& \textit{ResponseFacade resp , } \\
			& \textit{MultiPartObject mpo , }\\
			& \textit{ModelInformation mi , }\\
			& \textit{LoginableUser u )} \\
			\addlinespace
			& \textit{JSONObject getFormConfig( ModelInformation mi , }
\\
			& \textit{RequestFacade req , } \\
			& \textit{LoginableUser u )} \\
			\addlinespace
			& \textit{\textbf{(optional)} String getItemText()} \\
			\addlinespace
			& \textit{\textbf{(optional)} String
getItemIconPath()}\\
			\addlinespace
			\textbf{Graphical representation} & \textit{a single
menu item or context button} \\
			\textbf{Example(s)} & \textit{Behavioral Interface
Generation,} \\
& \textit{Deploy to SolutionCenter} \\
			\bottomrule
			\end{tabular}
		\end{table}
	
	\subsection{Details}
		\subsubsection{Basic Information}
			As we can derive from the superclass name
\textit{FormServerPlugin}, this type of plug-in displays a form. As an advantage over simple plug-ins the entered form
data can be used while processing the users request. It is recommended to use this plug-in type, when the displayed form has to be generated dynamically. If the general shape of the form is indepent of a concrete model, or model state, please use \textit{DialogPlugin} as superclass (cf. Section \ref{dialog_plugins}).
			
			When the form is sent back to the server it is delivered
in a special format called multipart. For details on multipart see Section
\ref{multipart_format}. The implementer of the plug-in does not have to care
about parsing this format since that is done before the respective methods are
called.
		
		\subsubsection{Methods}
			\paragraph{void performAction(RequestFacade req,
ResponseFacade resp, MultiPartObject mpo, ModelInformation mi, LoginableUser u
);}
			After the form has been submitted, this method is responsible for processing the incoming
request and sending a resulting response to the requester.
			All available information is contained within the
\textit{ModelInformation} and the \textit{MultiPartObject} objects. The latter
contains the data the user entered into the form belonging to this plug-in. 
			
			See Section \ref{multipart_format} for information on
the multipart format.			
				
			See Section \ref{response_section} for information on
how responses must be structured.
			
			\paragraph{JSONObject getFormConfig(ModelInformation mi, RequestFacade req, LoginableUser lu)}
			Return a \textit{JSONObject} that configures the required form.
To facilitate this step there exist several classes within the package
\verb!com.inubit.research.! \verb!server.extjs!. Detailed information on these classes is
given in Section \ref{extjs}. Two buttons, "Submit" and "Cancel" are
automatically added to the form.
			
			\paragraph{String getItemText();}
			Return the text for the menu item. If the
\verb!showInToolbar()!-method for this plug-in returns \verb!true! the item text
will be used as tooltip text instead.

			\paragraph{String getItemIconPath();}
			Return the path to the icon of the menu item. This is
especially important if the plug-in is represented as a simple button within the
toolbar or as a context menu button.
	
		
		\subsubsection{ExtJS-Form Creation}
		\label{extjs}
		The creation of ExtJS form configurations is facilitated by the
\textit{ExtJSFormFactory} class. By calling 
		\verb!createEmptyForm()! you will receive an empty form to which
you can add all required items. As all objects returned by the factory extend the class \textit{JSONObject}, no further conversion is required.
		
		Supported ExtJS form elements are (further elements may be implemented):
		\begin{itemize}
			\item Container elements (can have multiple
sub-elements):
			\begin{itemize}
				\item FieldSet
				\item CheckboxGroup
			\end{itemize}
			\item Simple elements:
			\begin{itemize}
				\item Checkbox
				\item TextField
			\end{itemize}
		\end{itemize}
		
		For configuring these elements (and also the form itself) use
the respective \verb!setProperty! \verb!(key, value)! method. To view all
configurable attributes take a look at
\url{http://www.extjs.com/deploy/dev/docs/}.
		
		For accessing the entered data during request processing, you
have to specify the \textbf{name}-attribute for each simple element. This name
can then be used to get the corresponding value out of the created
\textit{MultiPartObject} instance.
			
		\subsubsection{Multipart Format}
		\label{multipart_format}
		When the user submits the form, the server receives the data in
multipart format. In a first step this format is transferred into a Java object
of type \textit{MultiPartObject}. Accessing this object will deliver the entered
data to the plug-in. 
		
		The following methods are considered to be helpful, where
\verb!mpo! is an instance of class \textit{MultiPartObject}, \verb!mi! is a
\textit{MultiPartItem}, and \verb!mp! is a \textit{SimpleMultipartParser}:
		\begin{itemize}
			\item \verb!mbo.getItems()!\\
			This returns all items contained in the multipart
object. That means, that all none empty (or unset) form elements are returned by
this call.
			\item \verb!mbo.getItemByName(String name)!\\
			Return one specific item that is identified by its
unique name. The name is taken from the \textbf{name}-attribute of the form
element. The form element is only found if the name exists and the element has a
none-\verb!null! value. For checkbox elements this means, that they are only
part of the submitted form if they were checked when submitting the form.
			\item \verb!mbi.getContent()!\\
			Get the textual content of a form element. This is equal
to the value the user entered into the specific field.
		
			\item
\verb!sp.parseItemContentAsByteArray(BufferedInputStream bis, String itemName)!\\
			Reads a specific item of the input stream as byte array.
This can be used, e.g., to parse an image
			
		\end{itemize}