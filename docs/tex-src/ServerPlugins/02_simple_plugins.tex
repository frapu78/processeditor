\section{Simple Plug-Ins}
	\subsection{Overview}
		\begin{table}[ht]
			\centering
			\begin{tabular}{@{}lr@{}}
			\toprule
			\textbf{Required superclass} &
\textit{SimpleServerPlugin} \\
			\textbf{Configurable Sub-URLs} & \textit{no} \\
			\textbf{Requires JavaScript} & \textit{no} \\
			\textbf{Methods to implement} & \textit{void
performAction( ModelInformation mi , }\\
			& \textit{RequestFacade req , } \\
			& \textit{ResponseFacade resp , } \\
			& \textit{LoginableUser u )} \\
			\addlinespace
			& \textit{String getItemText()} \\
			\addlinespace
			& \textit{\textbf{(optional)} String
getItemIconPath()}\\
			\addlinespace
			\textbf{Graphical representation} & \textit{a single
menu item or context button} \\
			\textbf{Example(s)} & \textit{Layout} \\
			\bottomrule
			\end{tabular}
		\end{table}
	
	\subsection{Details}
		\subsubsection{Basic Information}
		Simple plug-ins are, as their name is already telling, the
simplest way of implementing a plug-in. Their graphical representation within
the editor's interface is limited to a single menu item within the
"Plugins"-menu, or directly within the editor's toolbar. The latter can be
achieved by overriding the \textit{showInToolbar()} method.
		
		If the user selects the menu item, a POST-request is generated
and sent to the plug-in's unique URL. The POST data consists of the current
model's ID along with the IDs of selected nodes and edges. In a first step this
JSON data is transformed into a \textit{ModelInformation} object which is then
passed to the \verb!performAction! method of your simple plug-in.
		
		\subsubsection{Methods}
			\paragraph{void performAction( ModelInformation mi,
RequestFacade req, ResponseFacade resp, LoginableUser u );}
			This method is responsible for processing the incoming
request and sending a resulting response to the requester.
			All required information has to be taken from the
\textit{ModelInformation} object. 
			
			See Section \ref{response_section} for information on
how responses must be structured.
			
			\paragraph{String getItemText();}
			Return the text for the menu item. If the
\textit{showInToolbar()}-method for this plug-in returns \verb!true! the item
text will be used as tool-tip text instead.

			\paragraph{String getItemIconPath();}
			Return the path to the icon of the menu item. This is
especially important if the plug-in is represented as a simple button within the
toolbar or as a context menu button.
			
			